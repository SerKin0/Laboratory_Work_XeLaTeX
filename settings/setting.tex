% ТЕКСТ
\setmainfont{Times New Roman} % Семейство шрифта

% ЗАГАЛОВКИ
% Настройка нумерации секций и подсекций
\titleformat{\section}
{\normalfont\Large\bfseries}{\thesection}{0.5em}{}
\titleformat{\subsection}
{\normalfont\large\bfseries}{\thesubsection}{1em}{}
% Настройка заголовка section
\titlespacing*{\section}{0pt}{15pt}{10pt} % Отступы: слева, сверху, снизу
\titlespacing*{\subsubsection}{0pt}{0pt}{2pt} % Отступы: слева, сверху, снизу

% Переопределим вывод нумерации, чтобы для subsection было в формате "0.0"
\renewcommand{\thesubsection}{\thesection.\arabic{subsection}}
\renewcommand{\thesubsubsection}{\thesection.\arabic{subsubsection}}

% КОЛОНТУЛЫ
\usepackage{fancyhdr} % Подключение fancyhdr
\pagestyle{fancy} % Установка стиля fancy для всего документа

\fancyhf{} % Очистка текущих значений колонтитулов

% Верхний колонтул
\fancyhead[R]{\normalsize \mytitle} % Правый верхний колонтитул
\fancyhead[L]{\normalsize \myauthor} % Левый верхний колонтитул

\renewcommand{\headrulewidth}{1pt} % Толщина линии в верхнем колонтитуле

% Нижний колонтул
\fancyfoot[C]{\small \thepage} % Нумерация страниц снизу по центру
\renewcommand{\footrulewidth}{0pt} % Убираем линию в нижнем колонтитуле

% Уровень нумерации секций
\setcounter{secnumdepth}{4} % Глубина нумерации секций

% ПРОГРАММИРОВАНИЕ
\lstset{
	inputencoding=utf8,
	extendedchars=true,
	keepspaces=true,
	language=Python,
	basicstyle=\small\fontspec{JetBrains Mono}, % Задаём шрифт
	numbers=left,
	numberstyle=\scriptsize,
	stepnumber=1,
	numbersep=5pt,
	backgroundcolor=\color{white},
	showspaces=false,
	showstringspaces=false,
	showtabs=false,
	frame=single,
	tabsize=2,
	captionpos=t,
	breaklines=true,
	breakatwhitespace=false,
	escapeinside={\%*}{*)}
}

% ГРАФИКИ
\usepackage{pgfplots}
\usepackage{pgfplotstable}


% ОБЪЕКТЫ
\setlength{\columnsep}{20pt} % Устанавливаем расстояние между колонками
\newcommand{\intro}[1]{
	\stepcounter{section}
	\section*{\hfillПРИЛОЖЕНИЕ \arabic{section}}
	\begin{center}
		\bf{#1}
	\end{center}
	\markboth{\MakeUppercase{#1}}{}
	\addcontentsline{toc}{section}{Приложение \arabic{section}. #1}
	\refstepcounter{theAppend}\par\vspace{1.5cm plus 1cm minus .5cm} {\hfill\bfПРИЛОЖЕНИЕ~\arabic{theAppend}. }
	\begin{center} \bf\MakeUppercase{#1} \end{center}
	\addcontentsline{toc}{section}{Приложение \arabic{theAppend}. #1}
}
