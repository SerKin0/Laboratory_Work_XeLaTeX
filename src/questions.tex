\section{Ответы на вопросы}

\subsection{Вопрос 1}

Что вы можете сказать о величине и направлении скорости и ускорения груза в следующих трех положениях груза (): а) крайнее верхнее положение; б) переход через положение равновесия; в) крайнее нижнее положение.


\subsection{Вопрос 2}
Какие превращения энергии происходят при колебаниях груза?

\subsection{Вопрос 3}
Как изменится период колебаний, если отрезав часть пружины, сделать ее короче?

\subsection{Вопрос 4}
Получите уравнение гармонического осциллятора для груза на пружине из закона сохранения механической энергии.

\subsection{Вопрос 5}
Найдите $A$ и $\varphi$ в законе движения груза (\ref{eqution:2}) в трех различных случаях возбуждения колебаний: 1) Груз отпущен с высоты на $h$ выше положения недеформированной пружины; 2) На $h$ ниже того же положения недеформированной пружины; 3) Груз в положении равновесия получил скорость $v$;

\subsection{Вопрос 6}
Как будет зависеть период колебаний от амплитуды, если при достаточно больших амплитудах закон Гука нарушается?