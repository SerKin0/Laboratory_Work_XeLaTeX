\section{Вывод}


В ходе данного исследования было определено значение коэффициента жёсткости $k$ с использованием формулы \ref{eqution:1} путём измерения удлинения $\triangle l$ для каждой пружины при различных значениях массы груза $M$.

Были проведены измерения периодов колебаний для каждой пружины с разными грузами $M$, после чего были построены графики зависимости $T^2$ от $M$, которые были сопоставлены с теоретической зависимостью $T^2 = \frac{4 \pi^2 M}{k}$.

Следует отметить, что линия тренда не пересекает точку 0, следовательно, при $M = 0$ колебания будут иметь место.

Было установлено, что амплитуда колебаний не оказывает влияния на период колебаний с учётом погрешности измерений.

Также было замечено, что со временем колебания затухают по обратной экспоненциальной зависимости.
