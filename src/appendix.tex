\section{Приложение}
	\begin{enumerate}
		\item \subsubsection*{Вычисление коэффициента упругости}
		\label{appendix: 1}
		\begin{multicols}{2}
			\(\begin{aligned}k_1 = \frac{0,1 * 9,81}{0,405 - 0,365} = 24,52 \text{ Н/м}\end{aligned}\) \\\\
			\(\begin{aligned}k_2 = \frac{0,2 * 9,81}{0,447 - 0,365} = 23,93 \text{ Н/м}\end{aligned}\) \\\\
			\(\begin{aligned}k_3 = \frac{0,3 * 9,81}{0,489 - 0,365} = 23,73 \text{ Н/м}\end{aligned}\) \\\\
			\(\begin{aligned}k_4 = \frac{0,4 * 9,81}{0,53 - 0,365} = 23,78 \text{ Н/м}\end{aligned}\) \\\\
			\(\begin{aligned}k_5 = \frac{0,1 * 9,81}{0,285 - 0,26} = 39,24 \text{ Н/м}\end{aligned}\) \\\\
			\(\begin{aligned}k_6 = \frac{0,2 * 9,81}{0,312 - 0,26} = 37,73 \text{ Н/м}\end{aligned}\) \\\\
			\(\begin{aligned}k_7 = \frac{0,3 * 9,81}{0,336 - 0,26} = 38,72 \text{ Н/м}\end{aligned}\) \\\\
			\(\begin{aligned}k_8 = \frac{0,4 * 9,81}{0,36 - 0,26} = 39,24 \text{ Н/м}\end{aligned}\) \\\\
		\end{multicols}
		
		\item \subsubsection*{Вычисление приборной погрешности}
		\label{appendix: 6}
		\(\begin{aligned} \triangle l = t_{\alpha, \infty} * \frac{\delta _l}{3} = 1,96 * \frac{0,1 \text{ см}}{3} = 0,065 \text{ см} \end{aligned}\) \\\\
		\(\begin{aligned} \triangle l_0 = t_{\alpha, \infty} * \frac{\delta _l}{3} = 1,96 * \frac{0,1 \text{ см}}{3} = 0,065 \text{ см} \end{aligned}\) \\\\
		\(\begin{aligned} \triangle M = t_{\alpha, \infty} * \frac{\delta _M}{3} = 1,96 * \frac{1 \text{ г}}{3} = 0,65 \text{ г} \end{aligned}\) \\\\
		
		\item \subsubsection*{Вычисление косвенной погрешности коэффициента упругости}
		\label{appendix: 5}
		\(\begin{aligned}\triangle k_1 = \frac{9,81}{0,405 - 0,365} \sqrt{0,0007^2 + 0,1^2 \frac{0,0007^2 + 0,0007^2}{(0,405 - 0,365)^2}} = 0,59\end{aligned}\) \\
		\(\begin{aligned}\triangle k_2 = \frac{9,81}{0,447 - 0,365} \sqrt{0,0007^2 + 0,2^2 \frac{0,0007^2 + 0,0007^2}{(0,447 - 0,365)^2}} = 0,28\end{aligned}\) \\
		\(\begin{aligned}\triangle k_3 = \frac{9,81}{0,489 - 0,365} \sqrt{0,0007^2 + 0,3^2 \frac{0,0007^2 + 0,0007^2}{(0,489 - 0,365)^2}} = 0,18\end{aligned}\) \\
		\(\begin{aligned}\triangle k_4 = \frac{9,81}{0,53 - 0,365} \sqrt{0,0007^2 + 0,4^2 \frac{0,0007^2 + 0,0007^2}{(0,53 - 0,365)^2}} = 0,14\end{aligned}\) \\
		\(\begin{aligned}\triangle k_5 = \frac{9,81}{0,285 - 0,26} \sqrt{0,0007^2 + 0,1^2 \frac{0,0007^2 + 0,0007^2}{(0,285 - 0,26)^2}} = 1,47\end{aligned}\) \\
		\(\begin{aligned}\triangle k_6 = \frac{9,81}{0,312 - 0,26} \sqrt{0,0007^2 + 0,2^2 \frac{0,0007^2 + 0,0007^2}{(0,312 - 0,26)^2}} = 0,68\end{aligned}\) \\
		\(\begin{aligned}\triangle k_7 = \frac{9,81}{0,336 - 0,26} \sqrt{0,0007^2 + 0,3^2 \frac{0,0007^2 + 0,0007^2}{(0,336 - 0,26)^2}} = 0,48\end{aligned}\) \\
		\(\begin{aligned}\triangle k_8 = \frac{9,81}{0,36 - 0,26} \sqrt{0,0007^2 + 0,4^2 \frac{0,0007^2 + 0,0007^2}{(0,36 - 0,26)^2}} = 0,37\end{aligned}\) \\
		
		\item \subsubsection*{Расчет среднеквадратичного отклонения}
		\label{appendix: 2}
		\(\begin{aligned}S_{t 1} = \frac{\sqrt{(4,9 - 4,9)^2 + (4,9 - 4,9)^2 + (4,9 - 5,0)^2}}{6}\end{aligned}\) = 0,01 c\\
		\(\begin{aligned}S_{t 2} = \frac{\sqrt{(6,2 - 6,3)^2 + (6,2 - 6,2)^2 + (6,2 - 6,1)^2}}{6}\end{aligned}\) = 0,02 c\\
		\(\begin{aligned}S_{t 3} = \frac{\sqrt{(7,3 - 7,3)^2 + (7,3 - 7,4)^2 + (7,3 - 7,3)^2}}{6}\end{aligned}\) = 0,01 c\\ 
		\(\begin{aligned}S_{t 4} = \frac{\sqrt{(8,3 - 8,2)^2 + (8,3 - 8,3)^2 + (8,3 - 8,3)^2}}{6}\end{aligned}\) = 0,01 c\\
		\(\begin{aligned}S_{t 5} = \frac{\sqrt{(4,0 - 3,9)^2 + (4,0 - 4,1)^2 + (4,0 - 3,9)^2}}{6}\end{aligned}\) = 0,03 c\\ 
		\(\begin{aligned}S_{t 6} = \frac{\sqrt{(4,6 - 4,7)^2 + (4,6 - 4,5)^2 + (4,6 - 4,6)^2}}{6}\end{aligned}\) = 0,02 c\\
		\(\begin{aligned}S_{t 7} = \frac{\sqrt{(5,5 - 5,5)^2 + (5,5 - 5,6)^2 + (5,5 - 5,4)^2}}{6}\end{aligned}\) = 0,02 c\\
		\(\begin{aligned}S_{t 8} = \frac{\sqrt{(6,5 - 6,6)^2 + (6,5 - 6,5)^2 + (6,5 - 6,4)^2}}{6}\end{aligned}\) = 0,02 c\\
				
		\item \subsubsection*{Расчет случайной погрешности}
		\label{appendix: 3}
		\begin{multicols}{2}
			\(\begin{aligned}\triangle t_{\text{сл 1}} = 2,306 * 0,014 = 0,03 c\end{aligned}\) \\
			\(\begin{aligned}\triangle t_{\text{сл 2}} = 2,306 * 0,024 = 0,05 c\end{aligned}\) \\
			\(\begin{aligned}\triangle t_{\text{сл 3}} = 2,306 * 0,014 = 0,03 c\end{aligned}\) \\  
			\(\begin{aligned}\triangle t_{\text{сл 4}} = 2,306 * 0,014 = 0,03 c\end{aligned}\) \\
			\vfil
			\(\begin{aligned}\triangle t_{\text{сл 5}} = 2,306 * 0,027 = 0,06 c\end{aligned}\) \\ 
			\(\begin{aligned}\triangle t_{\text{сл 6}} = 2,306 * 0,024 = 0,05 c\end{aligned}\) \\ 
			\(\begin{aligned}\triangle t_{\text{сл 7}} = 2,306 * 0,024 = 0,05 c\end{aligned}\) \\
			\(\begin{aligned}\triangle t_{\text{сл 8}} = 2,306 * 0,024 = 0,05 c\end{aligned}\) \\ 
		\end{multicols}
		
		\item \subsubsection*{Расчет приборной погрешности}
		\label{appendix: 4}
		\(\begin{aligned}
			t_{\text{пр}} = 1,96 * \frac{0,5 c}{3} = 0,33 c
		\end{aligned}\)
		
		\item \subsubsection*{Вычисление косвенной погрешности времени}
		\label{appendix: 7}
		\begin{multicols}{2}
			\(\begin{aligned}\triangle t_1 = \sqrt{0,33^2 + 0,026^2} = 0,33\end{aligned}\) \\
			\(\begin{aligned}\triangle t_2 = \sqrt{0,33^2 + 0,042^2} = 0,33\end{aligned}\) \\
			\(\begin{aligned}\triangle t_3 = \sqrt{0,33^2 + 0,029^2} = 0,33\end{aligned}\) \\
			\(\begin{aligned}\triangle t_4 = \sqrt{0,33^2 + 0,032^2} = 0,33\end{aligned}\) \\
			\(\begin{aligned}\triangle t_5 = \sqrt{0,33^2 + 0,064^2} = 0,33\end{aligned}\) \\
			\(\begin{aligned}\triangle t_6 = \sqrt{0,33^2 + 0,046^2} = 0,33\end{aligned}\) \\
			\(\begin{aligned}\triangle t_7 = \sqrt{0,33^2 + 0,049^2} = 0,33\end{aligned}\) \\
			\(\begin{aligned}\triangle t_8 = \sqrt{0,33^2 + 0,051^2} = 0,33\end{aligned}\) \\
		\end{multicols}
		
		\item \subsubsection*{Вычисление косвенной погрешности периода колебаний}
		\label{appendix: 8}
		\begin{multicols}{2}
			\(\begin{aligned}\triangle T_1^* = \frac{2*0,49*0,33}{3} = 0,11 \text{ с}^2\end{aligned}\) \\\\
			\(\begin{aligned}\triangle T_2^* = \frac{2*0,62*0,33}{3} = 0,14 \text{ с}^2\end{aligned}\) \\\\
			\(\begin{aligned}\triangle T_3^* = \frac{2*0,73*0,33}{3} = 0,16 \text{ с}^2\end{aligned}\) \\\\
			\(\begin{aligned}\triangle T_4^* = \frac{2*0,83*0,33}{3} = 0,18 \text{ с}^2\end{aligned}\) \\\\
			\(\begin{aligned}\triangle T_5^* = \frac{2*0,4*0,33}{3} = 0,09 \text{ с}^2\end{aligned}\) \\\\
			\(\begin{aligned}\triangle T_6^* = \frac{2*0,46*0,33}{3} = 0,10 \text{ с}^2\end{aligned}\) \\\\
			\(\begin{aligned}\triangle T_7^* = \frac{2*0,55*0,33}{3} = 0,12 \text{ с}^2\end{aligned}\) \\\\
			\(\begin{aligned}\triangle T_8^* = \frac{2*0,65*0,33}{3} = 0,14 \text{ с}^2\end{aligned}\) \\\\
		\end{multicols}
		
		\item \subsubsection*{Вычисление квадрата периода}
		\label{appendix: 9}
		\begin{multicols}{2}
			\(\begin{aligned} T_1^* = \frac{4*\pi^2*0,1}{24,52} = 0,161 \text{ с}^2\end{aligned}\) \\\\
			\(\begin{aligned} T_2^* = \frac{4*\pi^2*0,2}{23,93} = 0,330 \text{ с}^2\end{aligned}\) \\\\
			\(\begin{aligned} T_3^* = \frac{4*\pi^2*0,3}{23,73} = 0,499 \text{ с}^2\end{aligned}\) \\\\
			\(\begin{aligned} T_4^* = \frac{4*\pi^2*0,4}{23,78} = 0,663 \text{ с}^2\end{aligned}\) \\\\
			\(\begin{aligned} T_5^* = \frac{4*\pi^2*0,1}{39,24} = 0,101 \text{ с}^2\end{aligned}\) \\\\
			\(\begin{aligned} T_6^* = \frac{4*\pi^2*0,2}{37,73} = 0,209 \text{ с}^2\end{aligned}\) \\\\
			\(\begin{aligned} T_7^* = \frac{4*\pi^2*0,3}{38,72} = 0,306 \text{ с}^2\end{aligned}\) \\\\
			\(\begin{aligned} T_8^* = \frac{4*\pi^2*0,4}{39,24} = 0,402 \text{ с}^2\end{aligned}\) \\\\
		\end{multicols}
		\item \subsubsection*{Вычисление косвенной погрешности квадрата периода}
		\label{appendix: 10}
		\(\begin{aligned}\triangle T_1^* = \frac{4 \pi^2 \sqrt{0,59^2 * 24,52^2 + 0,59^2 * 0,1^2}}{24,52^2} = 0,004 \text{ с}^2\end{aligned}\) \\\\
		\(\begin{aligned}\triangle T_2^* = \frac{4 \pi^2 \sqrt{0,28^2 * 23,93^2 + 0,28^2 * 0,2^2}}{23,93^2} = 0,004 \text{ с}^2\end{aligned}\) \\\\
		\(\begin{aligned}\triangle T_3^* = \frac{4 \pi^2 \sqrt{0,18^2 * 23,73^2 + 0,18^2 * 0,3^2}}{23,73^2} = 0,004 \text{ с}^2\end{aligned}\) \\\\
		\(\begin{aligned}\triangle T_4^* = \frac{4 \pi^2 \sqrt{0,14^2 * 23,78^2 + 0,14^2 * 0,4^2}}{23,78^2} = 0,004 \text{ с}^2\end{aligned}\) \\\\
		\(\begin{aligned}\triangle T_5^* = \frac{4 \pi^2 \sqrt{1,47^2 * 39,24^2 + 1,5^2 * 0,1^2}}{39,24^2} = 0,004 \text{ с}^2\end{aligned}\) \\\\
		\(\begin{aligned}\triangle T_6^* = \frac{4 \pi^2 \sqrt{0,68^2 * 37,73^2 + 0,68^2 * 0,2^2}}{37,73^2} = 0,004 \text{ с}^2\end{aligned}\) \\\\
		\(\begin{aligned}\triangle T_7^* = \frac{4 \pi^2 \sqrt{0,48^2 * 38,72^2 + 0,48^2 * 0,3^2}}{38,72^2} = 0,004 \text{ с}^2\end{aligned}\) \\\\
		\(\begin{aligned}\triangle T_8^* = \frac{4 \pi^2 \sqrt{0,37^2 * 39,24^2 + 0,37^2 * 0,4^2}}{39,24^2} = 0,004 \text{ с}^2\end{aligned}\) \\\\
		\item \subsubsection*{Вычисление периода колебаний}
		\label{appendix: 11}
		\begin{multicols}{3}
			\(\begin{aligned}T_1 = \frac{7,6}{10} = 0,76 \text{ с}\end{aligned}\) \\\\
			\(\begin{aligned}T_2 = \frac{7,4}{10} = 0,74 \text{ с}\end{aligned}\) \\\\
			\(\begin{aligned}T_3 = \frac{7,2}{10} = 0,72 \text{ с}\end{aligned}\) \\\\
			\(\begin{aligned}T_4 = \frac{5,3}{10} = 0,53 \text{ с}\end{aligned}\) \\\\
			\(\begin{aligned}T_5 = \frac{5,2}{10} = 0,52 \text{ с}\end{aligned}\) \\\\
			\(\begin{aligned}T_6 = \frac{5,1}{10} = 0,51 \text{ с}\end{aligned}\) \\\\
		\end{multicols}
		
		\item \subsubsection*{Расчет среднеквадратичного отклонения}
		\label{appendix: 12}
		\(\begin{aligned}S_{t 1} = \frac{\sqrt{(0,0 - 0,0)^2 + (0,0 - 0,0)^2 + (0,0 - 0,0)^2}}{6} = 0,00 \text{ с} \end{aligned}\) \\\\
		\(\begin{aligned}S_{t 2} = \frac{\sqrt{(78,0 - 70,0)^2 + (78,0 - 56,0)^2 + (78,0 - 108,0)^2}}{6} = 6,34 \text{ с} \end{aligned}\) \\\\ 
		\(\begin{aligned}S_{t 3} = \frac{\sqrt{(200,3 - 183,0)^2 + (200,3 - 180,0)^2 + (200,3 - 238,0)^2}}{6} = 7,70 \text{ с} \end{aligned}\) \\\\ 
		\(\begin{aligned}S_{t 4} = \frac{\sqrt{(538,3 - 545,0)^2 + (538,3 - 515,0)^2 + (538,3 - 555,0)^2}}{6} = 4,91 \text{ с} \end{aligned}\) \\\\ 
		\(\begin{aligned}S_{t 5} = \frac{\sqrt{(0,0 - 0,0)^2 + (0,0 - 0,0)^2 + (0,0 - 0,0)^2}}{6} = 0,00 \text{ с} \end{aligned}\) \\\\ 
		\(\begin{aligned}S_{t 6} = \frac{\sqrt{(9,7 - 10,0)^2 + (9,7 - 10,0)^2 + (9,7 - 9,0)^2}}{6} = 0,14 \text{ с} \end{aligned}\) \\\\ 
		\(\begin{aligned}S_{t 7} = \frac{\sqrt{(20,7 - 20,0)^2 + (20,7 - 22,0)^2 + (20,7 - 20,0)^2}}{6} = 0,27 \text{ с} \end{aligned}\) \\\\ 
		\(\begin{aligned}S_{t 8} = \frac{\sqrt{(59,7 - 69,0)^2 + (59,7 - 55,0)^2 + (59,7 - 55,0)^2}}{6} = 1,91 \text{ с} \end{aligned}\) \\\\
		
		\item \subsubsection*{Расчет случайной погрешности}
		\label{appendix: 13}
		\begin{multicols}{2}
			\(\begin{aligned}\triangle t_{\text{сл 1}} = 3,355 * 0,00 = 0,00 \text{ с}\end{aligned}\) \\ 
			\(\begin{aligned}\triangle t_{\text{сл 2}} = 3,355 * 6,34 = 21,28 \text{ с}\end{aligned}\) \\ 
			\(\begin{aligned}\triangle t_{\text{сл 3}} = 3,355 * 7,70 = 25,82 \text{ с}\end{aligned}\) \\ 
			\(\begin{aligned}\triangle t_{\text{сл 4}} = 3,355 * 4,91 = 16,46 \text{ с}\end{aligned}\) \\
			\vfill 
			\(\begin{aligned}\triangle t_{\text{сл 5}} = 3,355 * 0,00 = 0,00 \text{ с}\end{aligned}\) \\ 
			\(\begin{aligned}\triangle t_{\text{сл 6}} = 3,355 * 0,14 = 0,46 \text{ с}\end{aligned}\) \\ 
			\(\begin{aligned}\triangle t_{\text{сл 7}} = 3,355 * 0,27 = 0,91 \text{ с}\end{aligned}\) \\ 
			\(\begin{aligned}\triangle t_{\text{сл 8}} = 3,355 * 1,91 = 6,39 \text{ с}\end{aligned}\) \\
		\end{multicols}
		
		\item \subsubsection*{Расчет абсолютной погрешности}
		\label{appendix: 14}
		\begin{multicols}{2}
			\(\begin{aligned}\triangle t_1 = \sqrt{0,33^2 + 0,00^2} = 0,33 \text{ с}\end{aligned}\) \\
			\(\begin{aligned}\triangle t_2 = \sqrt{0,33^2 + 21,28^2} = 21,28 \text{ с}\end{aligned}\) \\
			\(\begin{aligned}\triangle t_3 = \sqrt{0,33^2 + 25,82^2} = 25,82 \text{ с}\end{aligned}\) \\
			\(\begin{aligned}\triangle t_4 = \sqrt{0,33^2 + 16,46^2} = 16,46 \text{ с}\end{aligned}\) \\
			\vfill
			\(\begin{aligned}\triangle t_5 = \sqrt{0,33^2 + 0,00^2} = 0,33 \text{ с}\end{aligned}\) \\
			\(\begin{aligned}\triangle t_6 = \sqrt{0,33^2 + 0,46^2} = 0,56 \text{ с}\end{aligned}\) \\
			\(\begin{aligned}\triangle t_7 = \sqrt{0,33^2 + 0,91^2} = 0,97 \text{ с}\end{aligned}\) \\
			\(\begin{aligned}\triangle t_8 = \sqrt{0,33^2 + 6,39^2} = 6,40 \text{ с}\end{aligned}\) \\
		\end{multicols}
	\end{enumerate}